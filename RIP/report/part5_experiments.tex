\section{experiments}

4 big experiments have been executed.

\begin{itemize}
	\item noise analysis
	\item tracking, manually changing the set points
	\item tracking using sine/stair functions
	\item disturbances
\end{itemize}


\subsection{noise analysis}


$\theta$ and $\alpha$ have been measured before and after the low pass filter. The difference between those signals is considered noise as displayed in figure~\ref{fig:time plot noise}. Instead of looking at the noise in function of time its very useful to look at a histogram as is done in figure~\ref{fig:hist noise}. The variance of the noise can easily be calculated and then used in the simulation.

\begin{figure}[H]
	\centering
	\begin{subfigure}[b]{0.45\textwidth}
		\includegraphics[width=\textwidth]{./part4_experiments/noise/simple_plot.png}
		\caption{time plot of $\theta$ and $\alpha$ before and after the low pass filter}
		\label{fig:time plot theta and alpha}
	\end{subfigure}
	\begin{subfigure}[b]{0.45\textwidth}
		\includegraphics[width=\textwidth]{./part4_experiments/noise/difference.png}
		\caption{Difference of $\theta$/$\alpha$ before and after the low pass filter}
		\label{fig:time plot noise}
	\end{subfigure}
	\begin{subfigure}[b]{0.45\textwidth}
		\includegraphics[width=\textwidth]{./part4_experiments/noise/histogram.png}
		\caption{Historgram of the noise}
		\label{fig:hist noise}
	\end{subfigure}
\end{figure}

\subsection{manual tracking}

\subsection{tracking}

\subsection{disturbances}